\documentclass{beamer}
\usetheme{Warsaw}
\useinnertheme{circles}
\useoutertheme[subsection=false]{smoothbars}
\usepackage[utf8x]{inputenc}
\usepackage[czech]{babel}
\usepackage[T1]{fontenc}
\usepackage{listings}
\usepackage{tikz}
\lstset{basicstyle=\tiny\ttfamily}
\logo{\includegraphics[height=0.5cm]{brmlab.pdf}}

\begin{document}

\AtBeginSection[]
{
  \begin{frame}
    \frametitle{Outline}
    \tableofcontents[currentsection]
  \end{frame}
}

\title{brmiversity: Umělá inteligence \\ a teoretická informatika}
\subtitle{Přednáška č. 12}
\author{Petr Baudiš $\langle${\tt pasky@ucw.cz}$\rangle$}
\institute{
	brmlab 2011\\
	\vskip 1ex
	\pgfdeclareimage[height=4ex]{ccbysa}{by-sa.pdf}
	\pgfuseimage{ccbysa}
}
\date{}
\frame{\titlepage}

\section{Umělá inteligence a adaptivní agenti}

\subsection{}
\begin{frame}{Reprezentace znalostí}
\begin{itemize}
\item Logická reprezentace vs. rámce vs. \dots
\item Syntax vs. sémantika
\item Výroková logika, SAT
\item Predikátová logika, temporální logika atd.
\item V adapt. agentech?
\vskip 3ex
\item Spíše strojové dokazování vět (další přednáška):
\item Odvozovací metody (model checking, inference rules --- )
\item Báze znalostí, redukce na PL (skolemizace a unifikace) a odvozování (liftování, grounding, dopřed./zpět. řetězení a rezoluce)
\item RETE, TODO další
\item V adapt. agentech?
\end{itemize}
\end{frame}

\subsection{}
\begin{frame}{Otázky?}
\begin{center}
Příště: Strojové dokazování vět (\dots ve znalostních bázích).
\end{center}
\end{frame}

\section{Neuronové sítě}

\subsection{}
\begin{frame}{Modulární, hierarchické a hybridní modely NN}
\begin{itemize}
\item RBF-sítě
\item Kaskádová korelace
\item Časové posloupnosti
\vskip 3ex
\item Adaptivní směsi NN
\item Sériová kompozice
\end{itemize}
\end{frame}

\subsection{}
\begin{frame}{Otázky?}
\begin{center}
Příště: Přehled aplikací NN.
\end{center}
\end{frame}

\section{Evoluční algoritmy}

\subsection{}
\begin{frame}{Aplikace}
\begin{itemize}
\item Řešení kombinatorických problémů
\item Neuroevoluce
\item Výběr akcí
\item Expertní systémy
\end{itemize}
\end{frame}

\subsection{}
\begin{frame}{Otázky?}
\begin{center}
To je o evolučních algoritmech vše!
\end{center}
\end{frame}

\section{Vyčíslitelnost}

\subsection{}
\begin{frame}{Algoritmicky nerozhodnutelné problémy}
\begin{itemize}
\item Již známe: Halting problem, Riceova věta
\item Další\dots
\end{itemize}
\end{frame}

\subsection{}
\begin{frame}{Otázky?}
\begin{center}
Příště: Gödelovy věty.
\end{center}
\end{frame}

\section{Složitost}

\subsection{}
\begin{frame}{Třídy složitosti}
\begin{itemize}
\item Rekapitulace --- P, NP, PSPACE, EXPTIME
\end{itemize}
\end{frame}

\subsection{}
\begin{frame}{Míry složitosti}
\begin{itemize}
\item Polynomiální hierarchie
\end{itemize}
\end{frame}

\subsection{}
\begin{frame}{Savičova věta}
\begin{itemize}
\item TODO
\end{itemize}
\end{frame}

\subsection{}
\begin{frame}{Konstruovatelné funkce}
\begin{itemize}
\item TODO
\end{itemize}
\end{frame}

\subsection{}
\begin{frame}{Věty o zrychlení a mezerách}
\begin{itemize}
\item TODO
\end{itemize}
\end{frame}

\subsection{}
\begin{frame}{Hierarchie tříd složitosti}
\begin{itemize}
\item TODO
\end{itemize}
\end{frame}

\subsection{}
\begin{frame}{Otázky?}
\begin{center}
Příště: Pseudopolynomiální a aproximační algoritmy.
\end{center}
\end{frame}

\subsection{}
\begin{frame}{Děkuji vám}
\begin{center}
{\bf pasky@ucw.cz}

\vskip 6ex

Příště: Umělá inteligence a adaptivní agenti. \\
	Neuronové sítě.
	Složitost. \\
	Datové struktury (v externí paměti).
\end{center}
\end{frame}

\end{document}
