\documentclass{beamer}
\usetheme{Warsaw}
\useinnertheme{circles}
\useoutertheme[subsection=false]{smoothbars}
\usepackage[utf8x]{inputenc}
\usepackage[czech]{babel}
\usepackage[T1]{fontenc}
\usepackage{listings}
\usepackage{tikz}
\lstset{basicstyle=\tiny\ttfamily}
\logo{\includegraphics[height=0.5cm]{brmlab.pdf}}

\begin{document}

\AtBeginSection[]
{
  \begin{frame}
    \frametitle{Outline}
    \tableofcontents[currentsection]
  \end{frame}
}

\title{brmiversity: Umělá inteligence \\ a teoretická informatika}
\subtitle{Přednáška č. 12}
\author{Petr Baudiš $\langle${\tt pasky@ucw.cz}$\rangle$}
\institute{
	brmlab 2011\\
	\vskip 1ex
	\pgfdeclareimage[height=4ex]{ccbysa}{by-sa.pdf}
	\pgfuseimage{ccbysa}
}
\date{}
\frame{\titlepage}

\section{Umělá inteligence a adaptivní agenti}

\subsection{}
\begin{frame}{Reprezentace znalostí}
\begin{itemize}
\item Logická reprezentace vs. rámce vs. \dots
\item Syntax vs. sémantika
\item Výroková logika, SAT
\item Predikátová logika, temporální logika atd.
\item V adapt. agentech?
\vskip 3ex
\item Spíše strojové dokazování vět (další přednáška):
\item Odvozovací metody (model checking, inference rules --- )
\item Báze znalostí, redukce na PL (skolemizace a unifikace) a odvozování (liftování, grounding, dopřed./zpět. řetězení a rezoluce)
\item RETE, TODO další
\item V adapt. agentech?
\end{itemize}
\end{frame}

\subsection{}
\begin{frame}{Otázky?}
\begin{center}
Příště: Strojové dokazování vět (\dots ve znalostních bázích).
\end{center}
\end{frame}

\section{Neuronové sítě}

\subsection{}
\begin{frame}{Modulární, hierarchické a hybridní modely NN}
\begin{itemize}
\item RBF-sítě
\item Kaskádová korelace
\item Časové posloupnosti
\vskip 3ex
\item Adaptivní směsi NN
\item Sériová kompozice
\end{itemize}
\end{frame}

\subsection{}
\begin{frame}{Otázky?}
\begin{center}
Příště: Přehled aplikací NN.
\end{center}
\end{frame}

\section{Evoluční algoritmy}

\subsection{}
\begin{frame}{Aplikace}
\begin{itemize}
\item Řešení kombinatorických problémů
\item Neuroevoluce
\item Výběr akcí
\item Expertní systémy
\end{itemize}
\end{frame}

\subsection{}
\begin{frame}{Otázky?}
\begin{center}
To je o evolučních algoritmech vše!
\end{center}
\end{frame}

\section{Složitost}

\subsection{}
\begin{frame}{Rekapitulace --- Turingův stroj}
\begin{itemize}
\item {\bf Turingův stroj:} Pětice $(Q, \Gamma, b, \Sigma, q_0, F, \delta)$
\item Nekonečná páska {\em (data)} rozdělená na buňky, v každé jedno písmeno z ``abecedy'';
	můžeme mít i více pásek!
\item Hlava stroje {\em (pozice na pásce)} se hýbe v jednom kroku o~buňku doleva nebo doprava
\item Stav stroje {\em (pozice v programu)} z množiny stavů
\item Přechodová funkce {\em (program)} podle stavu a písmena pod hlavou přepne stroj do nového stavu, zapíše nové písmeno a~posune hlavu
\end{itemize}
\end{frame}

\subsection{}
\begin{frame}{Rekapitulace --- Třídy složitosti}
\begin{itemize}
\item {\bf Třídy složitosti:} Skupina algoritmů se stejnou řádovou složitostí v závislosti na délce vstupu (P, NP, LSPACE, PSPACE, EXPTIME, \dots)
\item P: Všechny algoritmy, které běží v {\em polynomiálním čase} na~``obyčejném'' Turingově stroji (DTS)
\item NP: Algoritmy, které běží v polynomiálním čase na~nedeterministickém Turingově stroji (NTS)
\item PSPACE: Všechny algoritmy, které sežerou \\ {\em polynomiálně mnoho pásky}
\end{itemize}
\end{frame}

\subsection{}
\begin{frame}{PSPACE-úplnost}
\begin{itemize}
\item True Quantified Boolean Formula (TQBF) --- SAT s~posloupností kvantifikátorů
\item Hra na hlavní města
\item Poziční hry (hex, schody v Go, \dots)
\item ``Stromové'' hry: Mohou trvat dlouho, EXPTIME!
\end{itemize}
\end{frame}

\subsection{}
\begin{frame}{Vyčíslitelnost a složitost}
\begin{itemize}
\item Dnes: Trochu spojíme vyčíslitelnost a složitost
\item Dnes: Budeme zkoumat, jak se třídy složitosti \\ vzájemně proplétají
\vskip 3ex
\item Zjednodušení: Zadání i výsledek budeme kódovat $n$-ticí jedniček; zajímá nás nějaká $f\colon \mathbb{N} \to \mathbb{N}$
\item $f$ je {\bf rekurzivní}, pokud existuje DTS $M$ přepisující $1^n \to 1^{f(n)}$
\item $f$ je {\bf vyčíslitelná} v čase $O(f)$, pokud $M$ udělá \\ nejvýše $cf(n)$ kroků % c>=1, f rekurzivni
\item $f$ je {\bf vyčíslitelná} v prostoru $O(f)$, pokud $M$ použije \\ nejvýše $cf(n)$ buněk % c>=1, f rekurzivni
\end{itemize}
\end{frame}

\subsection{}
\begin{frame}{Konstruovatelné funkce}
\begin{itemize}
\item Jaké druhy funkcí můžou definovat složitost algoritmu?
\item $g$ je {\bf časově konstruovatelná}, pokud $M$ zastaví \\ právě po $g(n)$ krocích % g(n) > n
\item $g$ je {\bf prostorově konstruovatelná}, pokud $M$ použije \\ právě $g(n)$ buněk % g(n) > n
\item Lze dokázat: Pro superlineární funkce je už vyčíslitelnost a~konstruovatelnost to samé!
\end{itemize}
\end{frame}

\subsection{}
\begin{frame}{Hierarchie tříd složitosti}
\begin{center}
$DTIME(T(n)), DSPACE(S(n)), NTIME(T(n)), NSPACE(T(n))$

Třída jazyků (predikátů, problémů) řešitelných v určité mezi na~DTS resp. NTS
\end{center}
\begin{itemize}
\item $\forall n: F_1(n) \le F_2(n) \Rightarrow CLASS(F_1(n)) \subseteq CLASS(F_2(n))$
\item $DTIME(T(n)) \subseteq NTIME(T(n))$
\item $DTIME(T(n)) \subseteq DSPACE(T(n))$, $NTIME(T(n)) \subseteq DSPACE(T(n))$
\item $DSPACE(S(n)) \subseteq DTIME(c^{S(n)}) \qquad S(n) \ge \log_2(n)$
\item $NTIME(T(n)) \subseteq DTIME(c^{T(n)})$
\vskip 3ex
\item Časová a prostorová hierarchie jsou otevřené shora: Pro každou rekurzivní funkci $T$ existuje $L \notin DTIME(T(n))$.
\end{itemize}
\end{frame}

\subsection{}
\begin{frame}{Věty o zrychlení a mezerách}
\begin{itemize}
\item Věta o lineárním zrychlení DTS: Tím, že velmi rozšíříme množinu stavů, dokážeme lineárně zrychlovat práci stroje
\pause
\item Borodinova věta o mezerách: Pro každou rekurzivní funkci $g(n) \ge n$ existuje rekurzivní $T(n)$ (mez) taková, že $DTIME(T(n)) = DTIME(g(T(n)))$ \\ Tedy i pro velmi rychle rostoucí $g(n)$ existuje třída složitosti $T$, která ji pozře!
\pause
\item Blumova věta o zrychlení: Pro každou rekurzivní funkci $g$ (zrychlovací funkce) existuje rekurzivní predikát $f$ (úloha) takový, že pro každý DTS $M_i$ existuje DTS $M_j$ řešící $f$ tak, \\ že $g(T_j(n)) \le T_i(n)$ pro {\em skoro každý} vstup. \\ Řešení některých úloh tedy lze stále zrychlovat!
\vskip 2ex
\pause
\item Poučení: V teorii složitosti záleží jen na řádových změnách ($O$-notace má smysl)
\end{itemize}
\end{frame}

\subsection{}
\begin{frame}{Savičova věta}
\begin{itemize}
\item $S(n)$ prostorově konstruovatelná funkce, $S(n) \ge \log_2(n)$
\item {\bf Savičova věta:} $NSPACE(S(n)) \subseteq DSPACE(S^2(n))$
\item Prostorové omezení je mnohem volnější než časové! $PSPACE=NPSPACE$
\end{itemize}
\begin{block}{Idea důkazu}
\begin{itemize}
\item $TEST(I_1,I_2,i)$ --- ze stavu $I_1$ do $I_2$ za $2^i$ kroků?
\item $TEST(I_1,I_2,i)$ odpoví hned, může-li, jinak volá $TEST(I_1,I_x,i-1) \land TEST(I_x, I_2, i-1) \quad \forall x$
\item Spustím $TEST(I_0, I_p, kS(n))$ (stavů $2^{kS(n)}$)
\item Každý $TEST$ potřebuje $3S(n)$ paměti pro ukládání mezistavu, hloubka rekurze je $kS(n)$, tedy celkem $3kS^2(n)$
\end{itemize}
\end{block}
\end{frame}

\subsection{}
\begin{frame}{Orákulové systémy}
\begin{itemize}
\item {\bf Orákulum:} Black box, který ``umí něco řešit'' a radí vám (můžete jej použít při výpočtu)
\item V praxi neexistují! Užitečné při analýze algoritmů \\ (např. kryptografické algoritmy)
\item Notace: NP(C) je třída jazyků, které rozpozná NTS s~orákulem řešícím $C$
\vskip 3ex
\item NP(P) = \only<1>{?}\only<2->{NP}
\item P(P) = \only<-2>{?}\only<3->{P (zřetězíme výpočet)}
\item NP(NP) = \only<-3>{?}\only<4->{už nevíme}
\pause
\pause
\pause
\pause
\item Vždy platí $P(C) \subseteq NP(C) \subseteq PSPACE(C)$
\item Polynomiální hierarchie: $\Sigma_0 = P$, $\Sigma_{i+1} = NP(\Sigma_i)$, $PH = \bigcup_{i=0}^\infty \Sigma_i$
\item Platí $PH \subseteq PSPACE$ (indukcí podle $i$)
\end{itemize}
\end{frame}

\subsection{}
\begin{frame}{Otázky?}
\begin{center}
Příště: Pseudopolynomiální a aproximační algoritmy.
\end{center}
\end{frame}

\section{Vyčíslitelnost}

\subsection{}
\begin{frame}{Algoritmicky nerozhodnutelné problémy}
\begin{itemize}
\item Definice: Problém je algoritmicky rozhodnutelný, je-li jeho predikát rekurzivní. Jazyk, který není algoritmicky rozhodnutelný, je algoritmicky nerozhodnutelný.
\vskip 3ex
\item Již známe: Halting problem, Riceova věta
\item Obecný {\em rozhodovací problém} pravdivosti logických výroků
\item Je jazyk přijímaný daným TS prázdný, neprázdný, konečný?
\item Je daná TS vítězný přičinlivý bobr? (busy beaver champion)
\item Kolmogorovská složitost řetězce
\item Postův korespondenční problém
\item Řešení diofantických rovnic
\item Problém smrtelné matice
\item Kachlíkování nekonečné roviny
\end{itemize}
\end{frame}

\subsection{}
\begin{frame}{Otázky?}
\begin{center}
Příště: Gödelovy věty.
\end{center}
\end{frame}

\subsection{}
\begin{frame}{Děkuji vám}
\begin{center}
{\bf pasky@ucw.cz}

\vskip 6ex

Příště: Umělá inteligence a adaptivní agenti. \\
	Neuronové sítě.
	Složitost. \\
	Datové struktury (v externí paměti).
\end{center}
\end{frame}

\end{document}
