\documentclass{beamer}
\usetheme{Warsaw}
\useinnertheme{circles}
\useoutertheme[subsection=false]{smoothbars}
\usepackage[utf8x]{inputenc}
\usepackage[czech]{babel}
\usepackage[T1]{fontenc}
\usepackage{listings}
\usepackage{tikz}
\lstset{basicstyle=\tiny\ttfamily}
\logo{\includegraphics[height=0.5cm]{brmlab.pdf}}

\begin{document}

\AtBeginSection[]
{
  \begin{frame}
    \frametitle{Outline}
    \tableofcontents[currentsection]
  \end{frame}
}

\title{brmiversity: Umělá inteligence \\ a teoretická informatika}
\subtitle{Přednáška č. 6}
\author{Petr Baudiš $\langle${\tt pasky@ucw.cz}$\rangle$}
\institute{
	brmlab 2011\\
	\vskip 1ex
	\pgfdeclareimage[height=4ex]{ccbysa}{by-sa.pdf}
	\pgfuseimage{ccbysa}
}
\date{}
\frame{\titlepage}

\section{Pravděpodobnost}

\subsection{}
\begin{frame}{Pravděpodobnost}
\begin{itemize}
\item Co je to
\item Dvě interpretace
\item Pozor na fuzzy logiku
\item Operace s pravděpodobnostmi
\item Podmíněná pravděpodobnost
\item Bayesovo pravidlo
\item Měření pravděpodobnosti, střední hodnota, směrodatná odchylka a rozptyl
\item Pravděpodobnostní rozdělení
\item Normální rozdělení, intervaly spolehlivosti
\end{itemize}
\end{frame}

\section{Umělá inteligence}

\subsection{}
\begin{frame}{Zpracování neurčité informace}
\begin{itemize}
\item Přehled --- Usuzování, rozhodování, modelování, učení
\item Bayesovské sítě
\item Rozhodovací grafy
\end{itemize}
\end{frame}

\subsection{}
\begin{frame}{Otázky?}
\begin{center}
Příště UI: Modelování --- Markovské modely, Kalmanův filtr.
\end{center}
\end{frame}

\section{Složitost}

\subsection{}
\begin{frame}{Pravděpodobnostní algoritmy}
\begin{itemize}
\item Monte Carlo a Las Vegas algoritmy
\end{itemize}
\end{frame}

\subsection{}
\begin{frame}{Otázky?}
\begin{center}
Příště: Míry složitosti, Savitchova věta, konstruovatelné funkce.
\end{center}
\end{frame}

\section{Datové struktury}

\subsection{}
\begin{frame}{Hashování}
\begin{itemize}
\item Interní hashování, různé organizace hashovacích tabulek.
\item Externí hashování.
\item Očekávaný počet kolizí, složitosti.
\item Intro --- perfektní a univerzální hashování.
\end{itemize}
\end{frame}

\subsection{}
\begin{frame}{Otázky?}
\begin{center}
Příště: Univerzální a perfektní hashování. \\ A někdy doděláme ty haldy.
\end{center}
\end{frame}

\subsection{}
\begin{frame}{Děkuji vám}
\begin{center}
{\bf pasky@ucw.cz}

\vskip 6ex

Příště: Umělá inteligence. \\ Neuronové sítě (statistické zpracování dat). \\ Adaptivní agenti (komunikace a znalosti). Datové struktury.
\end{center}
\end{frame}

\end{document}
