\documentclass{beamer}
\usetheme{Warsaw}
\useinnertheme{circles}
\useoutertheme[subsection=false]{smoothbars}
\usepackage[utf8x]{inputenc}
\usepackage[czech]{babel}
\usepackage[T1]{fontenc}
\usepackage{listings}
\usepackage{tikz}
\lstset{basicstyle=\tiny\ttfamily}
\logo{\includegraphics[height=0.5cm]{brmlab.pdf}}

\begin{document}

\AtBeginSection[]
{
  \begin{frame}
    \frametitle{Outline}
    \tableofcontents[currentsection]
  \end{frame}
}

\title{brmiversity: Umělá inteligence \\ a teoretická informatika}
\subtitle{Přednáška č. 13}
\author{Petr Baudiš $\langle${\tt pasky@ucw.cz}$\rangle$}
\institute{
	brmlab 2011\\
	\vskip 1ex
	\pgfdeclareimage[height=4ex]{ccbysa}{by-sa.pdf}
	\pgfuseimage{ccbysa}
}
\date{}
\frame{\titlepage}

\section{Umělá inteligence a adaptivní agenti}

\subsection{}
\begin{frame}{Reprezentace znalostí}
\begin{itemize}
\item Logická reprezentace vs. rámce vs. \dots
\item Syntax vs. sémantika
\item Výroková logika, SAT
\item Predikátová logika, temporální logika atd.
\item V adapt. agentech?
\end{itemize}
\end{frame}

\begin{frame}{Strojové dokazování vět}
\begin{itemize}
\item Prvotní motivace: Automatické dokazování vět ve formálních systémech
\item Vlastně spíše: Odvozování logických důsledků v bázi znalostí
\vskip 3ex
\item Odvozovací metody (model checking, inference rules --- )
\item Báze znalostí, redukce na PL (skolemizace a unifikace) a odvozování (liftování, grounding, dopřed./zpět. řetězení a rezoluce)
\item RETE, TODO další
\item V adapt. agentech?
\end{itemize}
\end{frame}

\subsection{}
\begin{frame}{Otázky?}
\begin{center}
Příště: Splňování omezujících podmínek a automatické plánování.
\end{center}
\end{frame}

\section{Neuronové sítě}

\subsection{}
\begin{frame}{Modulární, hierarchické a hybridní modely NN}
\begin{itemize}
\item RBF-sítě
\item Kaskádová korelace
\item Časové posloupnosti
\vskip 3ex
\item Adaptivní směsi NN
\item Sériová kompozice
\end{itemize}
\end{frame}

\subsection{}
\begin{frame}{Otázky?}
\begin{center}
Příště: (Stručný!) přehled aplikací NN.
\end{center}
\end{frame}

\section{Složitost}

\subsection{}
\begin{frame}{Rekapitulace --- Třídy složitosti}
\begin{itemize}
\item {\bf Třídy složitosti:} Skupina algoritmů se stejnou řádovou složitostí v závislosti na délce vstupu (P, NP, LSPACE, PSPACE, EXPTIME, \dots)
\item P: Všechny algoritmy, které běží v {\em polynomiálním čase} na~``obyčejném'' Turingově stroji (DTS)
\item NP: Algoritmy, které běží v polynomiálním čase na~nedeterministickém Turingově stroji (NTS)
\item PSPACE: Všechny algoritmy, které sežerou \\ {\em polynomiálně mnoho pásky}
\vskip 3ex
\item NP-úplný problém\pause: Je v NP a lze na něj všechny NP problémy převést v polynomiálním čase
\end{itemize}
\end{frame}

\subsection{}
\begin{frame}{Pseudopolynomiální algoritmy}
\begin{itemize}
\item Kódování vstupu --- unární vs. jiné
\item Pseudopolynomiální algoritmus
\item Příklad (součet podmnožiny, další?)
\item Silná NP úplnost
\end{itemize}
\end{frame}

\subsection{}
\begin{frame}{Aproximační algoritmy}
\begin{itemize}
\item Míry aproximace
\item Aproximační schéma
\item ÚPAS
\item Příklad (TSP, další?)
\end{itemize}
\end{frame}

\subsection{}
\begin{frame}{Otázky?}
\begin{center}
To je o výpočetní složitosti vše!

\vskip 3ex

{\bf Nemluvili jsme o řadě krásných věcí \dots}

\vskip 1ex

TS s přímým přístupem, RAM stroje, booleovské obvody.

\vskip 1ex

L a NL, interaktivní dokazovací systémy, \\ třídy pravděpodobnostních algoritmů, \\
praktický nedeterminismus (kvantové počítače).
\end{center}
\end{frame}

\section{Datové struktury}

\subsection{}
\begin{frame}{Datové struktury ve vnější paměti}
\begin{itemize}
\item Vnější paměť: Pomalá, {\em blokový} přenos (po stránkách)
\vskip 3ex
\item B-sort
\item Externí hashování (tři druhy)
\item Další databázové techniky (mřížky atd.)
\end{itemize}
\end{frame}

\subsection{}
\begin{frame}{Otázky?}
\begin{center}
Příště: Třídění ve vnitřní a vnější paměti.
\end{center}
\end{frame}

\subsection{}
\begin{frame}{Děkuji vám}
\begin{center}
{\bf pasky@ucw.cz}

\vskip 6ex

Příště: Umělá inteligence.
	Neuronové sítě.
	Datové struktury.
	Vyčíslitelnost (Gödelovy věty).
\end{center}
\end{frame}

\end{document}
