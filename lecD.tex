\documentclass{beamer}
\usetheme{Warsaw}
\useinnertheme{circles}
\useoutertheme[subsection=false]{smoothbars}
\usepackage[utf8x]{inputenc}
\usepackage[czech]{babel}
\usepackage[T1]{fontenc}
\usepackage{listings}
\usepackage{tikz}
\lstset{basicstyle=\tiny\ttfamily}
\logo{\includegraphics[height=0.5cm]{brmlab.pdf}}

\begin{document}

\AtBeginSection[]
{
  \begin{frame}
    \frametitle{Outline}
    \tableofcontents[currentsection]
  \end{frame}
}

\title{brmiversity: Umělá inteligence \\ a teoretická informatika}
\subtitle{Přednáška č. 13}
\author{Petr Baudiš $\langle${\tt pasky@ucw.cz}$\rangle$}
\institute{
	brmlab 2011\\
	\vskip 1ex
	\pgfdeclareimage[height=4ex]{ccbysa}{by-sa.pdf}
	\pgfuseimage{ccbysa}
}
\date{}
\frame{\titlepage}

\section{Umělá inteligence a adaptivní agenti}

\subsection{}
\begin{frame}{Reprezentace znalostí}
\begin{itemize}
\item Doteď spíš emergentní, automaticky naučené znalosti
\item Dnes explicitní logická reprezentace
\item Výroková logika, SAT
\item Predikátová logika, temporální logika atd.
\vskip 3ex
\item Prvotní motivace: Dokazování vět ve formálních systémech
\item Teď: Odvozování logických důsledků v bázi znalostí --- {\em uvažování}
\vskip 3ex
\item Popis (jednoduchého) světa a logikou řízený agent
\item Deduktivní databáze a produkční systémy, \\ logické programování
\item Formální verifikace (software, hardware, \dots)
\item Strojové dokazování vět (matematická tvrzení, \\ automatický vědec)
\end{itemize}
\end{frame}

\subsection{}
\begin{frame}{Znalostní báze}
\begin{itemize}
\item Znalosti o světě i možné akce můžeme vyjádřit formálně \\ (třeba {\em logické predikáty} --- logické výrazy s proměnnými)
\item Zajímá nás, jak implementovat {\bf bázi znalostí} \\ (množinu všeho, co víme)
\item Operace $TELL$ přidá do báze nový poznatek
\item Operace $ASK$ bázi položí {\em dotaz} \\ Rozdíl vůči databázi: Testujeme splnitelnost logického výrazu, báze musí sama {\em odvodit důkaz}
\item Typicky nevíme vše --- neúplná informace, vzhledem ke~znalostem máme několik {\em možných modelů} světa; \\ pak musíme použít další heuristiky
\end{itemize}
\end{frame}

\subsection{}
\begin{frame}{Výroková a predikátová logika}
\begin{itemize}
\item Proměnné: Booleovsky ohodnocené (true, false) popisují ``mapu světa''
\item Formule: Řetězení podformulí operátory $\land, \lor, \lnot$ popisuje ``zákony světa''
\item $a \Rightarrow b$ je $\lnot b \lor a$, $a \Leftrightarrow b$ je $(a \land b) \lor (\lnot a \land \lnot b)$
\vskip 3ex
\item {\bf Syntax:} Definuje, co vše je formule
\item {\bf Sémantika:} Říká, jestli je v daném světě (modelu) \\ formule pravdivá
\vskip 3ex
\item Predikátová logika (zatím odložíme stanou): funkce (výroba ``podproměnných'') a predikáty (vztahy mezi proměnnými)
\item Existenční $\exists$ a univerzální $\forall$ kvalifikátor 
\end{itemize}
\end{frame}

\subsection{}
\begin{frame}{Odvozovací metody}
\begin{itemize}
\item $x_0 = 0 \quad x_1 = 1 \quad x_2 = 0 \quad x_3 = 1 \quad x_4 = 1$
\item $x_0 \lor \lnot x_1 \lor x_2 \lor \lnot x_3 \lor x_4 \Rightarrow x_5$
\item Zajímá nás, jestli platí $x_3 \lor x_5$ \hskip 2em Odvozování: Modus ponens
\item {\bf Ověřování modelu:} Spočítáme $x_5$ a dosadíme
\item {\bf Odvozovací pravidla:} Odhalíme tautologii ($x_3 \lor \lnot x_3$)
\item Korektnost: Nikdy neschválíme neplatný důsledek
\item Úplnost: Vždy schválíme platný důsledek
\vskip 3ex
\item Věta je {\bf splnitelná} $\Leftrightarrow$ je pravdivá v nějakém modelu ($x_0 \lor x_1$)
\item Věta je {\bf nesplnitelná} $\Leftrightarrow$ nepravdivá v každém modelu ($x_1 \land x_5$)
\item Máme $ASK(KB, \alpha)$ (je $\alpha$ logický důsledek báze $KB$?); můžeme převést na testování splnitelnosti
\item {\bf Enumerace:} Generuj a testuj --- zkus všechny kombinace výrokových proměnných, existuje nějaká splňená v $KB$ i $\alpha$?
\end{itemize}
\end{frame}

\subsection{}
\begin{frame}{Příklad: Wumpus World}
\begin{itemize}
\item Nákres a pravidla
\item Příklad znalostí a dotazu
\item (Slajdy R. Bartáka.)
\end{itemize}
\end{frame}

\subsection{}
\begin{frame}{Predikátová logika}
\begin{itemize}
\item Predikátovou logiku lze redukovat na výrokovou
\item {\bf Grounding:} Za proměnné dosadíme všechny možné termy \\ a z nich pak vyrobíme výrokové proměnné
\item {\bf Skolemizace:} Výsledky existenčních kvantifikátorů nám \\ bude vyrábět Skolemova konstanta, resp. funkce
\vskip 1ex
\item Grounding generuje spoustu irelevantních výrazů, \\ odvozujme přímo v PL!
\end{itemize}
\end{frame}

\subsection{}
\begin{frame}{Odvozování v predikátové logice}
\begin{itemize}
\item {\bf Lifting:} Kvantifikátory substituujeme až v modus ponens
\item Jak najdeme substituci kvantifikátorů?
\end{itemize}
\begin{center}
Nechť $Knows(Karel, Jarda)$
$UNIFY(Knows(Karel, x), Knows(y, z)) = \{ x/Jarda, y/Karel, z/Jarda \}$
$UNIFY(Knows(Karel, x), Knows(y, z)) = \{ y/Karel, x/z \}$
\end{center}
\begin{itemize}
\item {\bf Unifikace:} Substiticí je více možných, chceme najít {\em nejobecnější unifikátor}
\item {\bf Standardizace stranou:} Proměnné z univerzálních kvalifikátorů nazveme unikátně
\end{itemize}
\end{frame}

\subsection{}
\begin{frame}{Konjuktivní normální forma}
\begin{center}
CNF: konjunkce klauzulí, klauzule je disjunkce proměnných
$$(x_0 \lor \lnot x_1) \land (\lnot x_2 \lor x_3 \lor \lnot x_4)$$

Všechny výrazy lze převést na CNF (i v predikátové logice):
$$ \forall x: (\forall y: Animal(y) \Rightarrow Loves(x, y)) \Rightarrow (\exists y: Loves(y, x))$$
$$ \forall x: (\lnot \forall y: (\lnot Animal(y) \lor Loves(x,y))) \lor (\exists y: Loves(y, x)) $$
$$ \forall x: (\exists y: (Animal(y) \land \lnot Loves(x, y))) \lor (\exists y: Loves(y, x)) $$
$$ \forall x: (\exists y: (Animal(y) \land \lnot Loves(x, y))) \lor (\exists z: Loves(z, x)) $$
$$ ((Animal(F(x)) \land \lnot Loves(x, F(x)))) \lor (Loves(G(x), x)) $$
$$ (Animal(F(x)) \lor Loves(G(x), x)) \land (\lnot Loves(x, F(x)) \lor Loves(G(x),x)) $$

Testování splnitelnosti CNF je stále NP-úplné (SAT)

Nejtěžší formule mají $klauzulí/proměnných=4.3$
\end{center}
\end{frame}

\subsection{}
\begin{frame}{Ověřování modelů}
\begin{center}
$ASK(KB, \alpha)$; už jsme zkoušeli enumerovat
\end{center}

\begin{block}{DPLL (Davis, Putnam, Logemann, Loveland)}
\begin{itemize}
\item Rekurzivní backtracking s heuristikami
\item Brzká terminace, známe-li hodnotu některých klauzulí
\item H. ryzí proměnné: dosadíme, mají-li $\forall$ výskyty stejné znaménko
\item H. jednotkové klauzule: dosadíme, vystupuje-li proměnná samostatně
\end{itemize}
\end{block}

\begin{block}{WalkSAT}
\begin{itemize}
\item Hill-climbing random walk (rychlý, není úplný)
\item Iterativně vybíráme neplatné klauzule, s pravděpodobností $p$ přepeneme náhodnou proměnnou, jinak proměnnou, která opraví co nejvíce klauzulí
\end{itemize}
\end{block}
\end{frame}

\begin{frame}{Obecná rezoluce}
\begin{itemize}
\item $ASK(KB, \alpha)$ pomocí odvozovacích pravidel \\ (ne dosazováním hodnot)
\item Obecná rezoluce: Důkaz sporem!
\item Začneme s $KB \land \lnot \alpha$, převedeme na CNF
\item Aplikujeme rezoluční pravidlo: $ (l_1 \lor l_2), (\lnot l_2 \lor l_3) | (l_1 \lor l_3) $
\item Tedy hledáme a sjednocujeme klauzule s komplementárním (vzájemně negovaným) výskytem proměnné
\item Pokud se dvě klauzule zcela vymlátí, dokázali jsme spor \\ (a platí $\alpha$)
\item Pokud už nelze vytvářet nové klauzule, spor není (a platí $\lnot \alpha$)
\end{itemize}
\end{frame}

\begin{frame}{Hornovské klauzule}
\begin{itemize}
\item {\bf Hornovská klauzule:} Klauzule speciálního tvaru, kde je maximálně jeden pozitivní literál
\item $C \land (\lnot B \lor A) \land (\lnot C \lor \lnot D \lor B)$
\item Jak $KB$ v takové formě může vzniknout? \pause \\ Implikace s jedním důsledkem!
\item $C \land (B \Rightarrow A) \land (C \land D \Rightarrow B)$
\item Logické programování (Prolog)
\vskip 3ex
\item Jednoduché (rychlé a srozumitelné) odvozování
\item Rozhodovací problém má složitost pouze $O(N)$
\end{itemize}
\end{frame}

\begin{frame}{Řetězení}
\begin{block}{Dopředné řetězení}
\begin{itemize}
\item Idea: Z faktů generujeme postupně důsledky, dokud nedokážeme, co jsme chtěli
\item Pro každou klauzuli máme čítač nesplněných předpokladů
\item Postupně bereme důsledky splněných klauzulí (čítač na nule) a~dekrementujeme čítače dalších klauzulí
\item {\bf Rete algoritmus:} Síť závislostí, inkrementální dokazování
\end{itemize}
\end{block}

\begin{block}{Zpětné řetězení}
\begin{itemize}
\item Idea: Hledáme, jaké předpoklady musíme splnit pro dokázání důsledků, dokud nejsou všechny splněné
\end{itemize}
\end{block}
\end{frame}

\begin{frame}{Reprezentace znalostí}
\begin{itemize}
\item Kde začít?
\item Příklad: Elektrické obvody
\item Slajdy R. Bartáka
\item Kategorie a taxonomie
\end{itemize}
\end{frame}

\subsection{}
\begin{frame}{Otázky?}
\begin{center}
Příště: Automatické plánování (dokazování v situačním kalkulu) a~splňování omezujících podmínek.
\end{center}
\end{frame}

% \section{Neuronové sítě}
% 
% \subsection{}
% \begin{frame}{Modulární, hierarchické a hybridní modely NN}
% \begin{itemize}
% \item RBF-sítě
% \item Kaskádová korelace
% \item Časové posloupnosti
% \vskip 3ex
% \item Adaptivní směsi NN
% \item Sériová kompozice
% \end{itemize}
% \end{frame}
% 
% \subsection{}
% \begin{frame}{Otázky?}
% \begin{center}
% Příště: (Stručný!) přehled aplikací NN.
% \end{center}
% \end{frame}

\section{Složitost}

\subsection{}
\begin{frame}{Rekapitulace --- Třídy složitosti}
\begin{itemize}
\item {\bf Třídy složitosti:} Skupina algoritmů se stejnou řádovou složitostí v závislosti na délce vstupu (P, NP, LSPACE, PSPACE, EXPTIME, \dots)
\item P: Všechny algoritmy, které běží v {\em polynomiálním čase} na~``obyčejném'' Turingově stroji (DTS)
\item NP: Algoritmy, které běží v polynomiálním čase na~nedeterministickém Turingově stroji (NTS)
\item PSPACE: Všechny algoritmy, které sežerou \\ {\em polynomiálně mnoho pásky}
\vskip 3ex
\item NP-úplný problém\pause: Je v NP a lze na něj všechny NP problémy převést v polynomiálním čase
\end{itemize}
\end{frame}

\subsection{}
\begin{frame}{Pseudopolynomiální algoritmy}
\begin{itemize}
\item Kódování vstupu může být unární nebo binární (či jiné)
\item Někdy to není jedno; časová složitost vzhledem k délce vstupu vs. k velikosti čísla na vstupu
\item Buď $n$ číslo na vstupu; $l = \log_2 n$ délka zakódování ($n \simeq 2^l$)
\item {\bf Pseudopolynomiální algoritmus} má dobu běhu omezen polynomem v $n$, ne v $l$
\vskip 3ex
\item Každý polynomiální algoritmus je i pseudopolynomiální
\item Naopak to neplatí! (Např. součet podmnožiny.)
\vskip 3ex
\item {\bf Číselný problém} je takový, kde není nutně $n \le p(l)$
\item Není-li NP-úplný problém číselný, nelze jej řešit pseudopolynomiálním algoritmem
\item Je-li NP-úplný problém číselný, může být ještě {\bf silně NP-úplný} a také jej tak nelze řešit (např. TSP)
\end{itemize}
\end{frame}

\subsection{}
\begin{frame}{Aproximační algoritmy}
\begin{itemize}
\item Povolme suboptimální řešení NP-těžkých optimalizačních problémů
\item Budeme chtít polynomiální čas a odhad kvality vzhledem k optimu
\item Odhad kvality: Poměrová chyba $A/O$ nebo relativní chyba $|A-O|/O$
\item Např. vrcholové pokrytí lze aproximovat s poměrovou chybou max. 2
\vskip 3ex
\item {\bf Aproximační schéma:} $AS_X(Y,e)$ řeší $X(Y)$ s max. relativní chybou $e$
\item {\bf Úplně polynomiální aproximační schéma:} AS časem omezené v polynomu $l(Y)$ i $e$
\item Pro silně NP-úplné úlohy ÚPAS neexistuje (pokud $P \ne NP$)
\end{itemize}
\end{frame}

\subsection{}
\begin{frame}{Otázky?}
\begin{center}
To je o výpočetní složitosti vše!

\vskip 3ex

{\bf Nemluvili jsme o řadě krásných věcí \dots}

\vskip 1ex

TS s přímým přístupem, RAM stroje, booleovské obvody.

\vskip 1ex

L a NL, interaktivní dokazovací systémy, \\ třídy pravděpodobnostních algoritmů, \\
praktický nedeterminismus (kvantové počítače).
\end{center}
\end{frame}

\section{Datové struktury}

\subsection{}
\begin{frame}{Datové struktury ve vnější paměti}
\begin{itemize}
\item Vnější paměť: Pomalá, {\em blokový} přenos (po stránkách)
\item Můžeme chtít měřit složitost v počtu I/O operací
\vskip 3ex
\item B-sort
\item Externí hashování (Fagin, Cormack, Larson-Kalja)
\end{itemize}
\end{frame}

\subsection{}
\begin{frame}{Externí hashování}
\begin{block}{Faginovo hashování}
\begin{itemize}
\item Parametr tabulky: Prvních $k$ bitů hashe
\item $2^k$ řádků v tabulce, každý obsahuje ukazatel na stránku s množinou prvků
\item Pokud se nějaká stránka naplní, rozštěpí se a zvýší se $k$
\end{itemize}
\end{block}
\begin{block}{Perfektní hashování}
\begin{itemize}
\item Cormack: Mám hashovací funkce $h(x), g(i, x)$
\item Adresář je hashovaný přes $h$ a obsahuje $i$ pro $g$, hlavní soubor adresovaný přes $g$
\item Larson-Kalja: Jen $h$, tabulka stránek se separátorem (horní mez $h$)
\end{itemize}
\end{block}
\end{frame}

\subsection{}
\begin{frame}{Otázky?}
\begin{center}
Příště: Třídění ve vnitřní a vnější paměti.
\end{center}
\end{frame}

\subsection{}
\begin{frame}{Děkuji vám}
\begin{center}
{\bf pasky@ucw.cz}

\vskip 6ex

Příště: Umělá inteligence.
	Neuronové sítě (zbytky).
	Datové struktury.
	Vyčíslitelnost (Gödelovy věty).
\end{center}
\end{frame}

\end{document}
