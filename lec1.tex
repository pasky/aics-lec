\documentclass{beamer}
\usetheme{Warsaw}
\useinnertheme{circles}
\useoutertheme[subsection=false]{smoothbars}
\usepackage[utf8x]{inputenc}
\usepackage[czech]{babel}
\usepackage[T1]{fontenc}
\usepackage{listings}
\logo{\includegraphics[height=0.5cm]{brmlab.pdf}}

\begin{document}

\AtBeginSection[]
{
  \begin{frame}
    \frametitle{Outline}
    \tableofcontents[currentsection]
  \end{frame}
}

\title{brmiversity: Umělá inteligence\\a teoretická informatika}
\subtitle{Úvodní přednáška}
\author{Petr Baudiš $\langle${\tt pasky@ucw.cz}$\rangle$}
\institute{
	brmlab 2011\\
	\vskip 1ex
	\pgfdeclareimage[height=4ex]{ccbysa}{by-sa.pdf}
	\pgfuseimage{ccbysa}
}
\date{}
\frame{\titlepage}

\section{Slovo úvodem}

\subsection{}
\begin{frame}{Formát}
\begin{itemize}
\item Cyklus 14 přednášek; dvě pauzy v listopadu
\item Každá přednáška vystřídá několik témat, někdy souvisejících méně a někdy více
\item Motivace: Příprava na Mgr. státnice z teoretické informatiky na MFF UK
\item Pouze letmý {\em teoretický} úvod do různých témat --- malý časový rozsah
\end{itemize}
\begin{center}
\url{http://brmlab.cz/event/aics}
\end{center}
\end{frame}

\section{Umělá inteligence}

\subsection{}
\begin{frame}{Umělá inteligence}
\begin{itemize}
% TODO obrazek
\item Naučit počítače býti chytrými!
\item Rozpoznávat, analyzovat, dedukovat, \dots chápat?
\item Zkoumat svět, pomáhat lidem, nahradit lidi, vládnout lidem. :-)
\item Amorfní vědní oblast spojující specializovanější obory.
\end{itemize}
\end{frame}

\subsection{}
\begin{frame}{Témata UI}
\begin{itemize}
\item Exaktní techniky --- prohledávání v grafech, reprezentace znalostí, strojové dokazování, splňování omezujících podmínek a automatické plánování
\item Pravděpodobnostní techniky --- Bayesovské sítě, Kalmanův filtr, markovské modely
\item Strojové učení --- Bayesovské modely, učení s učitelem a bez učitele % TODO
\item Herní algoritmy
\end{itemize}
\end{frame}

\section{Neuronové sítě}

\subsection{}
\begin{frame}{Neuronové sítě}
\begin{itemize}
% TODO obrazek
\item Speciální rozsáhlý obor umělé inteligence
\item Umělé neuronové sítě --- {\em inspirováno} mozkem (nebo spíše neurovými zauzleními)
\item Asi ``nejtvrdší'' matematika, my si udržíme nadhled
\end{itemize}
\end{frame}

\subsection{}
\begin{frame}{Neuronová témata}
\begin{itemize}
\item Předzpracování dat --- PCA, ``složitost'' dat % TODO
\item Klasické neuronové sítě --- perceptron, vícevrstvé sítě a jejich vlastnosti, modulární sítě
\item Asociativní paměti --- Hopfieldova síť, další techniky, netradiční aplikace
\item Samoorganizace --- klastrování, Kohonenovy mapy
\item Hrst příkladů aplikací
\end{itemize}
\end{frame}

\section{Adaptivní agenti}

\subsection{}
\begin{frame}{Adaptivní agenti}
\begin{itemize}
% TODO obrazek
\item Umělé bytosti --- umělá inteligence řídící chování autonomních agentů
\item Realistické nebo účelné chování, řízení postav, modelování emocí
\item Zábava --- počítačové hry!
\item Seriozní postavy --- tréninkové situace, terapeutické pomůcky
\item Biologie --- komputační etologie (zkoumání chování zvířat)
\end{itemize}
\end{frame}

\subsection{}
\begin{frame}{Témata pro adaptivní agenty}
\begin{itemize}
\item Architektury řízení agentů, metody a algoritmy
\item Navigace a hledání cesty v reálném prostředí
\item Komunikace a znalosti v multiagentních systémech
\item Úvod do komputační etologie
\end{itemize}
\end{frame}

\section{Evoluční algoritmy}

\subsection{}
\begin{frame}{Evoluční algoritmy}
\begin{itemize}
% TODO obrazek
\item Jak řešit problémy, které se nám nechce (nebo neumíme) programovat
\item Genetické algoritmy --- řešení algoritmů inspirované přírodním výběrem
\item Hledáme přibližná řešení těžkých problémů
\end{itemize}
\end{frame}

\subsection{}
\begin{frame}{Genetická témata}
\begin{itemize}
\item Základní genetický algoritmus, genetické operátory
\item Genetické a evoluční programování
\item Aplikace na reálné problémy
\item Teoretické modely genetických algoritmů
\end{itemize}
\end{frame}

\section{Složitost}

\subsection{}
\begin{frame}{Složitost}
\begin{itemize}
% TODO obrazek
\item Studium náročnosti algoritmů v závislosti na velikosti datasetu
\item Jak dlouho trvá algoritmus --- nezávisle na rychlosti počítače
\item Turingovy stroje, třídy složitosti, převody mezi stejně složitými problémy
\item Rozděl a panuj, dynamické programování, hladový algoritmus
\item Pravděpodobnostní a pseudopolynomiální algoritmy, aproximační algoritmy
\item Matematické nuance --- vztah mezi časovou a prostorovou složitostí
\end{itemize}
\end{frame}

\section{Datové struktury}

\subsection{}
\begin{frame}{Datové struktury}
\begin{itemize}
% TODO obrazek
\item Pokročilé datové struktury a jejich matematická analýza.
\item Haldy, hashování, hashování$^2$, binární stromy, B-stromy
\item Struktury ve vnější paměti
\end{itemize}
\end{frame}

\section{Vyčíslitelnost}

\subsection{}
\begin{frame}{Vyčíslitelnost}
\begin{itemize}
\item Co vůbec {\em umíme} spočítat? Které problémy rozhodnout neumíme?
\item O některých programech nevíme, jestli někdy skončí.
\item Jak matematicky popisovat a studovat programy a množiny jimy popisované?
\item Gödelovy věty --- o členství prvku v dostatečně složité množině nedokážeme vždy rozhodnout.
\end{itemize}
\end{frame}

\section{Základní datové struktury a algoritmy}

\subsection{}
\begin{frame}{Základní datové struktury}
\begin{itemize}
\item Pole --- prvky za sebou v paměti (hashe?)
\item Spojový seznam --- krabičky rozházené po paměti, každá obsahuje ukazatel na sousední
% TODO: obrazek
\item Graf --- sada vrcholů pospojovaných hranami $(V,E)\quad E \subseteq V\times V$
\item Strom --- graf bez ``cyklů''
\end{itemize}
\end{frame}

\subsection{}
\begin{frame}{Třídění}
\begin{itemize}
\item Jak uspořádat telefonní seznam?
\item Bubble sort
\item Quick sort
\item Merge sort
\end{itemize}
\end{frame}

\subsection{}
\begin{frame}{Děkuji vám}
\begin{center}
{\bf pasky@ucw.cz}
\end{center}
\end{frame}

\end{document}
