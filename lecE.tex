\documentclass{beamer}
\usetheme{Warsaw}
\useinnertheme{circles}
\useoutertheme[subsection=false]{smoothbars}
\usepackage[utf8x]{inputenc}
\usepackage[czech]{babel}
\usepackage[T1]{fontenc}
\usepackage{listings}
\usepackage{tikz}
\lstset{basicstyle=\tiny\ttfamily}
\logo{\includegraphics[height=0.5cm]{brmlab.pdf}}

\begin{document}

\AtBeginSection[]
{
  \begin{frame}
    \frametitle{Outline}
    \tableofcontents[currentsection]
  \end{frame}
}

\title{brmiversity: Umělá inteligence \\ a teoretická informatika}
\subtitle{Přednáška č. 14}
\author{Petr Baudiš $\langle${\tt pasky@ucw.cz}$\rangle$}
\institute{
	brmlab 2011\\
	\vskip 1ex
	\pgfdeclareimage[height=4ex]{ccbysa}{by-sa.pdf}
	\pgfuseimage{ccbysa}
}
\date{}
\frame{\titlepage}

\section{Umělá inteligence}

\subsection{}
\begin{frame}{Automatické plánování}
\begin{itemize}
\item Situační kalkulus
\item Rozvrhy
\end{itemize}
\end{frame}

\subsection{}
\begin{frame}{Constraint Satisfaction Problem}
\begin{itemize}
\item Splňování omezujících podmínek (CSP)
\item Domény
\end{itemize}
\end{frame}

\subsection{}
\begin{frame}{Otázky?}
\begin{center}
To je o umělé inteligenci vše!

\vskip 3ex

Samozřejmě mnoho restů \dots

Monte Carlo Markov Chains, Komputační lingvistika, Kálmanovy filtry pořádně, roboti v praxi, DSP\dots
\end{center}
\end{frame}

\section{Neuronové sítě}

\subsection{}
\begin{frame}{Modulární, hierarchické a hybridní modely NN}
\begin{itemize}
\item RBF-sítě
\item Kaskádová korelace
\item Časové posloupnosti
\vskip 3ex
\item Adaptivní směsi NN
\item Sériová kompozice
\end{itemize}
\end{frame}

\subsection{}
\begin{frame}{Aplikace NN}
\begin{itemize}
\item Klasifikátory
\item Prediktory
\item Výběr akce
\end{itemize}
\end{frame}

\subsection{}
\begin{frame}{Otázky?}
\begin{center}
To je vše!

\vskip 3ex

V současnosti nejpopulárnější variace jsou Support Vector Machines.
\end{center}
\end{frame}

\section{Datové struktury}

\subsection{}
\begin{frame}{Třídění}
\begin{itemize}
\item Bubblesort
\item Quicksort
\item Merge sort
\item Bucket sort
\item Funnel sort
\item B-sort?
\end{itemize}
\end{frame}

\subsection{}
\begin{frame}{Otázky?}
\begin{center}
To je vše.

Vícerozměrné struktury, dynamizace, TODO
\end{center}
\end{frame}

\section{Vyčíslitelnost}

\subsection{}
\begin{frame}{Gödelovy věty}
\begin{itemize}
\item Formulace
\item Důkaz
\vskip 3ex
\item A ve vyčíslitelnosti?
\end{itemize}
\end{frame}

\subsection{}
\begin{frame}{Otázky?}
\begin{center}
To je o výpočetní složitosti vše!

\vskip 3ex

Pominuli jsme mnoho dalších návazností: relativní vyčíslitelnost a aritmetickou hierarchii,
souvislost aritmetiky a vyčíslitelnosti a logiky, algoritmická náhodnost a Martingály, \dots
\end{center}
\end{frame}

\subsection{}
\begin{frame}{Děkuji vám}
\begin{center}
{\bf pasky@ucw.cz}

\vskip 6ex

Příště: už nic

brmiverzita (snad) nekončí!

Přihlašte se do {\bf announce@brmlab.cz} (pár mailů měsíčně).
\end{center}
\end{frame}

\end{document}
