\documentclass{beamer}
\usetheme{Warsaw}
\useinnertheme{circles}
\useoutertheme[subsection=false]{smoothbars}
\usepackage[utf8x]{inputenc}
\usepackage[czech]{babel}
\usepackage[T1]{fontenc}
\usepackage{listings}
\usepackage{tikz}
\lstset{basicstyle=\tiny\ttfamily}
\logo{\includegraphics[height=0.5cm]{brmlab.pdf}}

\begin{document}

\AtBeginSection[]
{
  \begin{frame}
    \frametitle{Outline}
    \tableofcontents[currentsection]
  \end{frame}
}

\title{brmiversity: Umělá inteligence\\a teoretická informatika}
\subtitle{Přednáška č. 3}
\author{Petr Baudiš $\langle${\tt pasky@ucw.cz}$\rangle$}
\institute{
	brmlab 2011\\
	\vskip 1ex
	\pgfdeclareimage[height=4ex]{ccbysa}{by-sa.pdf}
	\pgfuseimage{ccbysa}
}
\date{}
\frame{\titlepage}

\section{Adaptivní agenti}

\subsection{}
\begin{frame}{Agent v UI}
\begin{columns}
\begin{column}{6cm}
\begin{itemize}
\item TODO
\end{itemize}
\end{column}
\begin{column}{5cm}
% TODO obrazek
\end{column}
\end{columns}
\end{frame}

\subsection{}
\begin{frame}{Autonomní adaptivní agent (umělá bytost)}
\begin{columns}
\begin{column}{5cm}
% TODO obrazek
\end{column}
\begin{column}{6cm}
\begin{itemize}
\item TODO
\end{itemize}
\end{column}
\end{columns}
\end{frame}

\subsection{}
\begin{frame}{Architektura agenta}
\begin{itemize}
\item TODO
\end{itemize}
\end{frame}

\subsection{}
\begin{frame}{Otázky?}
\begin{center}
Příště: Pohyb agentů --- navigace v prostoru.
\end{center}
\end{frame}

\section{Evoluční algoritmy}

\subsection{}
\begin{frame}{Umělá evoluce}
\begin{columns}
\begin{column}{6cm}
\begin{itemize}
\item TODO
\end{itemize}
\end{column}
\begin{column}{5cm}
% TODO obrazek
\end{column}
\end{columns}
\end{frame}

\subsection{}
\begin{frame}{Genetický algoritmus}
\begin{columns}
\begin{column}{5cm}
% TODO obrazek
\end{column}
\begin{column}{6cm}
\begin{itemize}
\item TODO
\end{itemize}
\end{column}
\end{columns}
\end{frame}

\subsection{}
\begin{frame}{Genetické operátory}
\begin{itemize}
\item TODO
\end{itemize}
\end{frame}

\subsection{}
\begin{frame}{Příklad evolvovaného problému}
\begin{itemize}
\item TODO
\end{itemize}
\end{frame}

\subsection{}
\begin{frame}{Otázky?}
\begin{center}
Příště: Reprezentační schémata, evoluční a genetické programování.
\end{center}
\end{frame}

\section{Základní algoritmy}

\subsection{}
\begin{frame}{Minimální kostra grafu}
\end{frame}

\section{Datové struktury}

\subsection{}
\begin{frame}{Halda}
\begin{itemize}
\item TODO
\end{itemize}
\end{frame}

\subsection{}
\begin{frame}{Regulární halda}
\begin{itemize}
\item TODO
\end{itemize}
\end{frame}

\subsection{}
\begin{frame}{Pokročilé organizace hald}
\begin{itemize}
\item TODO
\end{itemize}
\end{frame}

\subsection{}
\begin{frame}{Levicové (leftist) haldy}
\begin{itemize}
\item TODO
\end{itemize}
\end{frame}

\subsection{}
\begin{frame}{Binomiální haldy}
\begin{itemize}
\item TODO
\end{itemize}
\end{frame}

\subsection{}
\begin{frame}{Fibonacciho haldy}
\begin{itemize}
\item TODO
\end{itemize}
\end{frame}

\subsection{}
\begin{frame}{Otázky?}
\begin{center}
Příště: Hashovací algoritmy.
\end{center}
\end{frame}

\section{Složitost}

\subsection{}
\begin{frame}{Rekapitulace: P a NP}
\begin{itemize}
\item TODO
\end{itemize}
\end{frame}

\subsection{}
\begin{frame}{Úplné problémy}
\begin{itemize}
\item TODO
\end{itemize}
\end{frame}

\subsection{}
\begin{frame}{Kachlíkování}
\begin{itemize}
\item TODO
\end{itemize}
\end{frame}

\subsection{}
\begin{frame}{3-SAT}
\begin{itemize}
\item TODO
\end{itemize}
\end{frame}

\subsection{}
\begin{frame}{Obchodní cestující}
\begin{itemize}
\item TODO
\end{itemize}
\end{frame}

\subsection{}
\begin{frame}{Hamiltonovská kružnice}
\begin{itemize}
\item TODO
\end{itemize}
\end{frame}

\subsection{}
\begin{frame}{Párování?}
\begin{itemize}
\item TODO
\end{itemize}
\end{frame}

\subsection{}
\begin{frame}{Součet podmnožiny (batoh)}
\begin{itemize}
\item TODO
\end{itemize}
\end{frame}

\subsection{}
\begin{frame}{Otázky?}
\begin{center}
Příště: Metody tvorby algoritmů.
\end{center}
\end{frame}

\subsection{}
\begin{frame}{Děkuji vám}
\begin{center}
{\bf pasky@ucw.cz}

Příště: Neuronové sítě, umělá inteligence (hry), \\ složitost, vyčíslitelnost (halting problem).
\end{center}
\end{frame}

\end{document}
