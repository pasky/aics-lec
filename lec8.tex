\documentclass{beamer}
\usetheme{Warsaw}
\useinnertheme{circles}
\useoutertheme[subsection=false]{smoothbars}
\usepackage[utf8x]{inputenc}
\usepackage[czech]{babel}
\usepackage[T1]{fontenc}
\usepackage{listings}
\usepackage{tikz}
\lstset{basicstyle=\tiny\ttfamily}
\logo{\includegraphics[height=0.5cm]{brmlab.pdf}}

\begin{document}

\AtBeginSection[]
{
  \begin{frame}
    \frametitle{Outline}
    \tableofcontents[currentsection]
  \end{frame}
}

\title{brmiversity: Umělá inteligence \\ a teoretická informatika}
\subtitle{Přednáška č. 8}
\author{Petr Baudiš $\langle${\tt pasky@ucw.cz}$\rangle$}
\institute{
	brmlab 2011\\
	\vskip 1ex
	\pgfdeclareimage[height=4ex]{ccbysa}{by-sa.pdf}
	\pgfuseimage{ccbysa}
}
\date{}
\frame{\titlepage}

\section{Adaptivní agenti}

\subsection{}
\begin{frame}{Metody pro řízení agentů}
\begin{itemize}
\item BDI, SOAR, atd.
\end{itemize}
\end{frame}

\subsection{}
\begin{frame}{Otázky?}
\begin{center}
Příště: Metody pro učení agentů.
\end{center}
\end{frame}

\section{Neuronové sítě}

\subsection{}
\begin{frame}{ANN Revisited}
\begin{itemize}
\item Umělé neurony (``výpočetní krabičky'') \\ dostávají vstupy (čísla) a na jejich \\ základě generují výstup (číslo)
\item Obvykle: Vrstvy striktně oddělené, \\ vstupní vrstva se vstupy zvnějšku, \\ výstupní vrstva s výstupem pro uživatele, \\ skryté vrstvy vyhodnocují různé charakteristiky vstupů
\item Dnes: Více vrstev neuronů, jak je učit?
\end{itemize}
\begin{tikzpicture}[remember picture,overlay]
  \node [xshift=-4.5cm,yshift=-6cm,above right] at (current page.north east)
    {\includegraphics[width=4cm]{ANN.pdf}};
\end{tikzpicture}
\end{frame}

\subsection{}
\begin{frame}{Backpropagation Revisited}
\begin{itemize}
\item Myšlenka: Závislosti mezi vstupy a výstupy dokážeme přiměřeně matematicky popsat
\item Chceme upravit váhy podle {\em chyby}, kterou propagovaly; \\ větší váha nese větší chybu
\vskip 3ex
\item Iterujeme učení podle vstupních množin:
\begin{itemize}
\item Zjistíme chybu výstupu
\item Spočítáme {\em gradient} chyby podle vah jednotlivých spojů
\item Chybu se pokusíme zredukovat posunutím vah proti gradientu
\item Chybu ``zpětně šíříme'' do předchozí vrstvy a opakujeme
\end{itemize}
\end{itemize}
\end{frame}

\subsection{}
\begin{frame}{Vylepšení učení ANN}
\begin{itemize}
\item Inicializace vah: rozumně malé, náhodné, vyvážené
\item Chytřejší sestup podle gradientu
\item Rozšíření gradientu derivacemi druhého řádu
\item Relaxační metody: perturbace vah
\item Modulární sítě, úprava parametrů (prahy, neurony)
\item Genetické algoritmy
\end{itemize}
\end{frame}

\subsection{}
\begin{frame}{BP s momentem}
\begin{center}
Setrvačnost (moment): Zamez oscilacím v úzkých údolích \\
(zejména jsou-li osy různě dlouhé)

$$\Delta_E w_{i,j}(t+1) = -\alpha{\partial E \over \partial w_{i,j}(t)} + \gamma(w_{i,j}(t)-w_{i,j}(t-1))$$

Najít $\alpha$ a $\gamma$ je magie \\
$\alpha$ musí být kompromis mezi lokálními minimy a oscilacemi
\end{center}
\end{frame}

\subsection{}
\begin{frame}{Dynamický parametr učení}
\begin{center}
$$\Delta_E w_{i,j}(t+1) = -\alpha_i{\partial E \over \partial w_{i,j}(t)}$$

V nelineárním případě je třeba $\alpha_i$ volit dynamicky

\begin{block}{Silva-Almeida}
\begin{itemize}
\item {\bf Urychluj}, pokud se nezměnilo znaménko; $\alpha_i' = u\alpha_i$, $u > 1$
\item {\bf Zpomaluj}, pokud se znaménko změnilo; $\alpha_i' = d\alpha_i$, $d < 1$
\item Exponenciální růst/pokles může být příliš velký
\end{itemize}
\end{block}

\begin{block}{Delta-bar-delta}
\begin{itemize}
\item Pokud se nezměnilo znaménko; $\alpha_i' = u+\alpha_i$,
\item Pokud se znaménko změnilo; $\alpha_i' = d\alpha_i$, $d < 1$
\item Změna znaménka oproti $\delta_i' = (1-\Phi)\Delta_iE' + \Phi\delta_i$
\end{itemize}
\end{block}
\end{center}
\end{frame}

\subsection{}
\begin{frame}{Super SAB}
\begin{itemize}
\item Kombinace dynamického $\alpha$ a momentu
\item Zpětné šíření s momentem; pokud se nezměnilo \\ znaménko derivace, $\alpha_i' = \alpha^+\alpha_i$
\item Pokud se změnilo znaménko derivace, undo, \\ zmenši parametr učení $\alpha_i' = \alpha^-\alpha_i$ a zkus znovu
\end{itemize}
\end{frame}

\subsection{}
\begin{frame}{Interní reprezentace znalostí}
\begin{itemize}
\item Jak posoudit efektivitu interní reprezentace (vah skrytých neuronů)?
\vskip 3ex
\pause
\item Chceme neurony, které jsou aktivní (výstup 1), \\ pasivní (výstup 0) nebo tiché (0.5) \\
	``Něco mezi'' tolik nepřispívá k jednoznačné klasifikaci
\item Jak byste udělali vzoreček?
\vskip 3ex
\pause
\item TODO
\vskip 3ex
\pause
\item Ještě lepší je mít neurony, které jsou buď aktivní nebo pasivní (nejsou nikdy zbytečné)
\pause
\item Interní reprezentace by měla být jednoznačná \\ (hodně odlišná pro hodně odlišné výstupy)
\end{itemize}
\end{frame}

\subsection{}
\begin{frame}{Prořezávání ANN}
\begin{itemize}
\item Zbavíme se neuronů s uniformní reprezentací
\item Zbavíme se neuronů s identickou či inverzní reprezentací \\ vůči jiným
\item Výslednou síť nazveme {\em redukovanou}, můžeme ji vždy vytvořit
\end{itemize}
\end{frame}

\subsection{}
\begin{frame}{Otázky?}
\begin{center}
Příště: ANN úplně jinak --- asociativní paměti.
\end{center}
\end{frame}

\section{Složitost}

\subsection{}
\begin{frame}{Třídy složitosti}
\begin{itemize}
\item Rekapitulace --- P, NP, PSPACE, EXPTIME
\end{itemize}
\end{frame}

\subsection{}
\begin{frame}{Míry složitosti}
\begin{itemize}
\item Polynomiální hierarchie
\end{itemize}
\end{frame}

\subsection{}
\begin{frame}{Savičova věta}
\begin{itemize}
\item TODO
\end{itemize}
\end{frame}

\subsection{}
\begin{frame}{Konstruovatelné funkce}
\begin{itemize}
\item TODO
\end{itemize}
\end{frame}

\subsection{}
\begin{frame}{Otázky?}
\begin{center}
Příště: Technické důsledky tříd složitosti.
\end{center}
\end{frame}

\section{Vyčíslitelnost}

\subsection{}
\begin{frame}{Věty o rekurzi}
\begin{itemize}
\item Prvá, druhá, třetí, divná\dots
\end{itemize}
\end{frame}

\subsection{}
\begin{frame}{Riceova věta}
\begin{itemize}
\item Nelze určit, zda dva programy dělají to samé
\item Důkaz
\end{itemize}
\end{frame}

\subsection{}
\begin{frame}{Otázky?}
\begin{center}
Příště: Algoritmicky nerozhodnutelné problémy.
\end{center}
\end{frame}

\subsection{}
\begin{frame}{Děkuji vám}
\begin{center}
{\bf pasky@ucw.cz}

\vskip 6ex

Příště: Umělá inteligence (strojové učení).
	Neuronové sítě. \\
	Evoluční algoritmy (schémata, EP, GP). \\
	Datové struktury (binární vyhledávací stromy).
\end{center}
\end{frame}

\end{document}
